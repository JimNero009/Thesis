\chapter*{Lay Summary}

\noindent

\normalsize

The startup of the Large Hadron Collider at CERN has marked the beginning of a new age of particle physics. We will have access to data about the most energetic particle collisions we have ever been able to produce. There are many hopes about what this data will reveal and a main one is that it will shed some light on how we can go beyond the Standard Model, which to date is our best theory of particle physics. The Standard Model has been incredibly successful but yet we know that it must be incomplete because it fails to explain, for example, the force of gravity. There are a number of potential ways we could extend the Standard Model and all of them predict either the existence of new particles that we have not yet seen or some modifications to the properties of the particles we already know about -- in many cases, both. In the latter case, these modifications can be very small and hard to detect and thus it is of paramount importance that we as a theory community provide accurate and precise predictions for how particles should behave within the bounds of the Standard Model. \\
\\
In this thesis, we argue the usual method of how these predictions are made is not satisfactory when describing the collisions at such a high energy. Instead, we should think about how the calculation is done in a slightly different way which guarantees that these high energy considerations are correctly taken into account. This is done in a fairly straightforward way for most of the potential collisions at the Large Hadron Collider, but requires more effort and thought when it is being expanded to include more possibilities. It is this latter point that constitutes new work on the part of the author. \\
\\
We show also that the formalism is simple enough that an approximation used in Higgs boson process no longer needs to be employed. This original result means we can therefore make predictions for this processes in a fundamentally new way. 

