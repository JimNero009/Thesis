\chapter{Abstract}

We discuss the properties of QCD amplitudes in the limit where all final state particles are well separated in rapidity. We study this limit in the context of the the High Energy Jets (HEJ) formalism. This formalism \emph{resums} terms in the perturbative expansion that go like $\log(\frac{s}{-t})$, which are enhanced in this limit. Such a region is reached already for modern day colliders such as the LHC and so capturing the contribution from these high energy logs is important in order to describe the data correctly. This is particularly important in, for example, VBF analyses where cuts are applied to pick out events with a large $m_{jj}$ and in many BSM searches. Following on from this, we discuss two directions in which HEJ's accuracy has been improved. \\
Firstly, we look at adding descriptions of partonic subprocesses which are formally sub-leading in the jet cross section but LL in the particular subprocess itself. This required the derivation of new \emph{effective vertices} that describe the emission of a quark/anti-quark pair in a way that is consistent with the resummation procedure. The inclusion of such processes means that HEJ is now less dependent on fixed-order calculations and that an important step towards full NLL accuracy has been achieved. \\
The second extension was to look at our description of events involving the emission of a Higgs boson along with jets. Specifically, we relax our requirement that the Higgs boson is produced with a rapidity that is far away from the rapidities of the accompanying jets and derive a new effective vertex from the study of the $qg \to qgH$ amplitude. Furthermore, the full dependence on the quark mass that appears in the loops connecting the gluons to the Higgs is kept, meaning that HEJ is unique in its ability to provide resummed predictions for this process with finite quark mass effects. The formalism is also simple enough to allow for extra final state jets in the process, a calculation that is currently beyond the reach of fixed order approaches. 