\chapter{Abstract}

The Large Hadron Collider (LHC) has provided, and will continue to provide, data for collisions at the highest energies ever seen in a particle accelerator. A strong knowledge of the properties of amplitudes for Quantum Chromodynamics in the High Energy Limit is therefore important to interpret this data. We study this limit in the context of the the High Energy Jets (HEJ) formalism. This formalism resums terms in the perturbative expansion of the cross section that behave like $\alpha_s^n \log(\frac{s}{-t})^{n-1}$, which are enhanced in this limit. Understanding this region is particularly important in certain key analyses at the LHC, for example, Higgs-boson-plus-dijet analyses where cuts are applied to pick out events with a large $m_{jj}$ and in many searches for new physics. \\
In this thesis, we discuss two directions in which HEJ's accuracy has been improved. Firstly, we look at adding descriptions of partonic subprocesses which are formally subleading in the jet cross section but Leading Log (LL) in the particular subprocess itself. This required the derivation of new effective vertices that describe the emission of a quark/anti-quark pair in a way that is consistent with the resummation procedure. The inclusion of such processes reduces HEJ's dependence on fixed-order calculations and marks an important step towards full Next-to-Leading Log (NLL) accuracy in the inclusive dijet cross section. \\
The second extension was to look at our description of events involving the emission of a Higgs boson along with jets. Specifically, we derive new effective vertices which keep the full dependence on the quark mass that appears in the loops the naturally arise in such amplitudes. The formalism is also simple enough to allow for extra final state jets in the process. Therefore, HEJ is unique in its ability to provide predictions for processes involving any number of final state jets accompanying a Higgs boson with the full finite quark mass effects kept, a calculation which is far beyond the reach of any fixed order approach. 

