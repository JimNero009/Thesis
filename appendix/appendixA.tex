\chapter{Spinor Algebra}
\label{app:spinor}

Since there are many points in this thesis where spinor algebra is used, we provide here a short review of the important properties of it. Our starting point is the Dirac equation;

\begin{equation}
(i \gamma^\mu \partial_\mu - m) \psi(x) = 0.
\end{equation}

The $\gamma^\mu$ are a set of four 4-by-4 matrices in spinor space and $\psi$ is a four-component spinor. In the derivation of this equation, we find that the gamma matrices are anti-commuting;

\begin{equation}
\{ \gamma^\mu, \gamma^\nu \} = 2 \eta^{\mu \nu}.
\end{equation}

This is the only restriction on these matrices and so there are a number of equivalent forms for them. We will use the following convention; 

\begin{subequations}
\begin{align}
\gamma^0 &= 
 \begin{pmatrix}
  0 & 0 & 1 & 0\\
  0 & 0 & 0 & 1\\
  1 & 0 & 0 & 0\\
  0 & 1 & 0 & 0
 \end{pmatrix},\\
 \gamma^1 &= 
 \begin{pmatrix}
  0 & 0 & 0 & -1\\
  0 & 0 & -1 & 0\\
  0 & 1 & 0 & 0\\
  1 & 0 & 0 & 0
 \end{pmatrix},\\
 \gamma^2 &= 
 \begin{pmatrix}
  0 & 0 & 0 & i\\
  0 & 0 & -i & 0\\
  0 & -i & 0 & 0\\
  i & 0 & 0 & 0
 \end{pmatrix},\\
 \gamma^3 &= 
 \begin{pmatrix}
  0 & 0 & -1 & 0\\
  0 & 0 & 0 & 1\\
  1 & 0 & 0 & 0\\
  0 & -1 & 0 & 0
 \end{pmatrix}.
 \end{align}
 \end{subequations}
 
We will also define the fifth gamma matrix $\gamma^5 = i \gamma^0 \gamma^1 \gamma^2 \gamma^3$. This finds use in the projection operator which can be applied to the Dirac spinor to pick out a certain helicity state; 

\begin{equation}
P_{\pm} = (1 \pm \gamma^5).
\end{equation}

In searching for plane wave solutions of the Dirac equation, we find four independent solutions;

\begin{subequations}
\begin{align}
u^{(1)} &= N
 \begin{pmatrix}
  1 \\
  0 \\
  \frac{p_z}{E+m} \\
  \frac{p_\perp}{E+m} 
 \end{pmatrix},\\
u^{(2)} &= N
 \begin{pmatrix}
  0 \\
  1 \\
  \frac{p_\perp^*}{E+m} \\
  \frac{-p_z}{E+m} 
 \end{pmatrix},\\
v^{(1)} &= N
 \begin{pmatrix}
   \frac{p_\perp^*}{E+m} \\
  \frac{-p_z}{E+m}  \\
 0\\
 1
 \end{pmatrix},\\
v^{(2)} &= -N
 \begin{pmatrix}
  \frac{p_z}{E+m} \\
  \frac{p_\perp}{E+m} \\
  1 \\
  0
 \end{pmatrix},
 \end{align}
 \end{subequations}

where $N$ is an overall normalisation used to fix the value of $u^\dagger u$ to anything that is desired. We will use the normalisation $u^\dagger u = 2E$ implicitly throughout this thesis. This is useful because the massless limit of these solutions can then be taken without any difficulty and indeed we will always take this limit. 

