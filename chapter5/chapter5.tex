\chapter{Conclusions}

In this thesis we have studied the perturbative approach to the solution of QCD scattering in the specific context of the Large Hadron Collider and discussed the limitations thereof. In chapter 1, we introduced the base theoretical knowledge needed to understand the technique, which included the discussion of gauge invariance and a general overview of the path integral formulation of Quantum Field Theory. This was used to summarise that calculations could be done by simply following a set of Feynman Rules and we explicitly calculated the $qQ \to qQ$ amplitude at leading order in $\alpha_s$ using these rules. We saw how this amplitude is converted into the physical cross section which can be measured at the collider and then re-did the calculation with spinor helicity formalism to show how it makes the amplitude calculations neater and quicker. \\
\\
In chapter 2, we explicitly showed how a truncation in the perturbative series is problematic when high energy particles are involved in the scattering. In particular, it was proved that terms that go like $\alpha_s^{n} \log^{n-1}\left(\frac{s}{-t} \right)$ appear at higher orders and, given their size, should not be neglected. This naturally led on to the discussion of Regge Theory and the formalism of High Energy Jets in general. The formalism exploits elements of the High Energy Limit in order to write amplitudes as simple current contractions over $t$-channel poles. By considering real and virtual corrections to processes in that limit, we ended up with a matrix element that can resum the problematic high energy logs to all orders in perturbation theory. In order to be of physical relevance, this matrix element is integrated over the relevant phase space via a Monte Carlo technique and the computational considerations of doing this are discussed. The chapter ends with a selection of distributions from real LHC analyses, where we show the capture of the high energy logs is already phenomenologically important and will only increase in importance as the centre of mass energy increases. \\
\\
In chapter 3, we discussed how the formalism can be extended to capture some of the Next-to-Leading Log contributions to jet processes, which will behave like $\alpha_s^{n} \log^{n-2}\left(\frac{s}{-t} \right)$. The addition of `unordered' gluon emissions allowed us to capture sub-leading contributions from already included partonic channels. The author's own work focused on the inclusion of matrix elements for entirely new partonic configurations which, although formally sub-leading in the jet process, are leading log in the particular subprocess. As such, the applicability of HEJ was increased and many checks of these new elements are presented along with a discussion of the computational challenges faced in including them within the Monte Carlo program. We concluded the chapter by investigating the effect of the NLL contributions on real data and showed that they provided a significant improvement to our predictions,. \\
\\
In chapter 4, we introduced how HEJ is also able to describe jet events accompanied by the emission of a Higgs boson. The effective theory where the mass of the top quark is taken to be infinite is discussed and its limitations laid out. The factorisation property that gives HEJ its resummation power does not rely on taking the infinite top mass limit and so amplitudes with full quark mass dependence were derived within the formalism. As a result, HEJ is unique in its ability to provide a prediction for such processes with both a high energy resummation and finite quark mass effects taken into account. By the presentation of a set of distributions, we saw how the effect of finite quark mass loops can lead to significantly different results; most drastically, in the behaviour of tails of the Higgs transverse momentum. \\
\\
In conclusion, the effects of large logarithmic contributions on QCD amplitudes are seen to be large and we must take them seriously in order to provide accurate Standard Model predictions. As the energy of colliders increase, these effects will only become more significant. It is therefore in the interests of both the HEJ collaboration and the phonological community in general to ensure an accurate inclusion of them. For HEJ, one aspect of this is to include some sub-leading partonic configurations in the resummation, since this reduces our dependence on leading order matching techniques. It remains a long term goal of the collaboration to include next-to-leading order matrix elements in the matching routine but the process is complicated by ensuring that no `double counting' occurs. Furthermore, the contributions of collinear terms are necessarily ignored by the approximations that underlie HEJ but must be considered for a complete description. This is currently done via an interface to the parton shower ARIADNE and there are plans to extend this to use the state-of-the-art parton shower of Pythia \cite{Sjostrand2007}. As well as the discussion of pure jet final states, HEJ also provides predictions for final states involving an electroweak boson. We now have the capability to move away from the currently implemented infinite top mass limit when discussing the emission of a Higgs boson. In order to detect any deviations in the discovered Higgs boson from the Standard Model predictions, it is vitally important that all known effects are correctly accounted for and this new capability of HEJ will help to a great degree towards that goal.